\subsection{Infrastructure}
\label{sec:infrastructure}

The experiments in this thesis were conducted using infrastructure suitable
to handle complex computations and large data sets.
This section will describe hardware (ch.~\ref{sub:meth_hardware}) and software
(ch.~\ref{sub:meth_software}) which was applied to the problem of tweet
engagement prediction.

\subsubsection{Hardware}
\label{sub:meth_hardware}

As stated in Chapter~\ref{sub:dl_drivers}, training deep neural networks
typically requires the parallel computing power of graphics processors.
In addition, large data files need to be stored and processed.
Cloud environments are well suited for these tasks, since they enable on-demand 
access to specialized hardware.
\textit{Amazon Web Services} was chosen as primary infrastructure provider for 
work on this thesis, mainly due to familiarity and a broad palette of offered
solutions.
Raw, unfiltered data files containing tweet and author information were stored
on an \textit{Amazon S3}\footnote{\url{https://aws.amazon.com/s3/}} instance,
which could be accessed programmatically or via the management console.
After filtering retweets and extracting features, the prepared data sets could
permanently be saved on the respective processing instance for improved
accessibility.
Training of deep learning models was undertaken on an \textit{Amazon EC2}\footnote{\url{https://aws.amazon.com/ec2/}}
instance.
Namely, the specification \textit{p2.xlarge} was chosen in order to train models
on a GPU.
These instances employ a \textit{NVIDIA K80} GPU with 12GB of memory and 2,496
processing cores.
Moreover, p2.xlarge machines possess 61GB of memory and four CPU cores.

\subsubsection{Software}
\label{sub:meth_software}

All model development was conducted using the \textit{Python}\footnote{\url{https://www.python.org/}} programming language,
since it offers extensive and well-docomented libraries for data analysis and numerical computation.
\textit{Jupyter notebooks}\footnote{\url{http://jupyter.org/}} were used as development environment, because
of their capability to combine code, visualizations and documentation in a single
document.
Conducting experiments in notebooks proved to increase reproducibility of results
and overall project organization.
Reusable functionality tested in notebooks was often transferred to Python
modules.
In order to collect data from the public Twitter API, a Python library called
\textit{python-twitter} was used.
The library provides wraps the API in that it directly converts the requested
data to Python objects.
This removed the need to deal with raw JSON data and made the data collection
process more convenient.
Several other Python libraries were applied to common tasks such as data
manipulation and analysis, namely \textit{pandas}\footnote{\url{https://pandas.pydata.org/}},
\textit{NumPy}\footnote{\url{http://www.numpy.org/}},
\textit{matplotlib}\footnote{\url{https://matplotlib.org/}} and
\textit{scikit-learn}\footnote{\url{http://scikit-learn.org/stable/}}.
Training deep neural networks effectively requires a multi-layered software
stack, which consists of low-level processing software and high-level libraries
for developing models conveniently.
In this thesis, \textit{Keras}\footnote{\url{https://keras.io/}} served as a
high-level API for developing and training models.
The library is compatible with several more low-level frameworks for numerical
computation, e.g., \textit{Theano}\footnote{\url{http://deeplearning.net/software/theano/}} 
or Google's \textit{TensorFlow}\footnote{\url{https://www.tensorflow.org/}}.
Out of these possible backends, Theano was used due to an easier setup procedure.
Processing the necessary computations for neural networks on graphics processors
further requires GPU-specific software.
Since a NVIDIA GPU was applied, the \textit{CUDA}\footnote{\url{https://developer.nvidia.com/cuda-zone}}
and \textit{cuDNN}\footnote{\url{https://developer.nvidia.com/cudnn}} toolkits
had to be installed.

This section about infrastructure concludes the methodology chapter.
The upcoming chapter will describe results of experiments undertaken in this
thesis.
