\section{Introduction}

\subsection{Problem}
\label{sec:problem}

\structure{Recent popularity of ML/DL}
\outline{Increase data availability for both individuals and companies}
\outline{Main sources: migration of transactions from the physical world to the
internet}
\outline{Examples: online shopping, banking, entertainment, customer support}
\outline{Result: more data, increased speed of access, more varied forms of data}
\outline{Organizations aim to make use of available data through data-driven
decision making}
\outline{Here, machine learning comes in, enables to recognize patters in data
that are useful to fulfill specific task}
\outline{With increasing computational resources, deep learning has emerged
as practical method, producing state-of-the-art results in many areas}
\outline{Example: computer vision, natural language processing, audio processing}

\structure{Recent popularity of social networks, esp., Twitter}
\outline{Social media platforms have revolutionized the way users interact on
the internet}
\outline{Enable users to create and share content, e.g., messages (Facebook, Twitter), 
images (Instagram, Snapchat), video (YouTube)}
\outline{Platforms make use of data, e.g., by allowing advertisers to target
specific market segments}
\outline{Social media has practical implications for organizations, since they
enable communication with and between customers}
\outline{Examples: social media campaigns, customer support, community building}
\outline{Twitter has emerged as microblogging platform which is used heavily
by individuals and organizations, e.g., for trend detection and breaking news}

\structure{Importance of engagement prediction - combining both fields}
\outline{Engagement prediction is where both fields come together}
\outline{In the context of Twitter, engagement prediction comprises estimating 
how often some piece of content will be retweeted or favorited by other users}
\outline{Possible use cases: anomaly detection (early detection of trends,
breaking news), optimization of content creation process (shaping content, time of
creation)}

\subsection{Objectives}
\label{sec:objectives}

\structure{Additional benefits from applying deep learning in contrast to prior work}
\outline{Forgo manual feature engineering, let model learn representation}
\outline{Use multi-class classification and regression to derive precise predictions}
\outline{Predict both retweets and favorites}
\outline{Learn advanced content features using modern NLP methods}
\outline{Apply models to diverse data sets for better generalization estimate}
\outline{Apply to user groups interested in this kind of model}
\outline{Find out about data requirements, apply to differently sized data sets}

\structure{Deployable model}
\outline{Minimal preprocessing}
\outline{Quick inference}
\outline{Ad-hoc prediction, not reliant on retweet cascades}

\subsection{Approach}
\label{sec:approach}

\structure{Data}
\outline{Build three different user groups - celebrities, politicians, companies}
\outline{Collect all tweets for given time frame as training data}
\outline{Data sets should employ diverse engagement distributions}
\outline{Additionally, construct bigger data set containing all user groups
and tweets from longer time frame}
\outline{For all data sets, extract structured features according to prior work
in this area}
\outline{Also, extract raw tweet texts as addtional, unstructured inputs}

\structure{Developed models}
\outline{First step: Start with linear baseline model as simplest model type}
\outline{Uses structured input features}
\outline{Second step: Develop deep feedforward neural networks, solely relying
on structured inputs as well}
\outline{Think of it as simplest form of neural network}
\outline{Third step: Develop more sophisticated multi-input neural network,
which takes both structured and unstructured inputs}
\outline{Compare results across model types and data sets}

