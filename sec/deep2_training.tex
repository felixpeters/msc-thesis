\subsubsection{Training process}
\label{sub:comb_training_process}

The previous section explained how the network architectures for experiments
in this chapter were chosen.
Before presenting results in the next section, the upcoming paragraphs will
detail characteristics of the training process.
In particular, special features of training more complex deep neural networks
are described.

Selected architecture were applied to all four data sets with the goal of
maximizing performance on unseen data.
Therefore, some examples were not used for training and instead used as validation
data.
The composition of the validation sets was identical to the ones used in previous
chapters.
As with previous models, adaptive moment estimation (Adam) with default settings
was used to optimize weights in the network.
Furthermore, no changes were applied to the cost functions.
Categorical cross-entropy was used for engagement classification, mean absolute
error was calculated when predicting singular engagement values.
Compared to simpler linear and deep feedforward networks, overfitting was
expected to become a more severe problem when training complex models.
This assumption is based on the larger number of parameters contained in these
networks.
With the embedding weights set to be trainable, the chosen regression model
with one convolutional layer consists of a total of 22,560,912 learnable parameters.
This high level of representational capability basically allows neural networks
to memorize smaller data sets.
This is contrary to the goal of developing a generalizable model.
Thus, the resulting challenge is to apply regularization and stop model training
when overfitting starts to occur.
Another expected observation was the increase in training time.
For example, running one epoch on the combined data set took about one minute
for a CNN and up to three minutes for models containing LSTM cells, compared
to about half a minute for deep feedforward networks.

Several techniques have been developed in the deep learning community to cope
with these special features of training deep neural networks.
Most of these best practices are implemented in deep learning software libraries
such as Keras.
For this work, \textbf{early stopping} and \textbf{model checkpoints} were applied
in order to manage longer training times and overfitting more effectively.
Early stopping denotes a process of monitoring the calculated validation loss,
minimization of which represents the main goal of model training.
If this metric does not decrease for a certain number of iterations, it is assumed
that a minimum was already reached and training is automatically stopped.
The accepted number of successive non-progress epochs is denoted as the
\textit{patience} hyperparameter, which has to be set prior to training.
For this work, the parameter was set from two to five epochs, depending on
model complexity.
Model checkpoints basically represent snapshots of the network state after training
for a specific number of epochs.
This state is represented by architecture and weight values.
Both of these state components can be saved for later access, e.g., in JSON~\footnote{\url{https://www.json.org/}} 
(more suitable for architecture) or HDF5~\footnote{\url{https://support.hdfgroup.org/HDF5/}}
(more suitable for weights) format.
In practice, this is most commonly done everytime a new validation loss minimum
was established.
Using both techniques in conjunction allows shortening training time and
repeated access to trained models.
Observations of experiments in this work were, that regression models took
more operations to minimize validation loss (up to 50 epochs) compared
to classification models (usually between ten and 20 epochs).
