\section{Conclusion}
\label{ch:conclusion}

This thesis applied deep learning methods to the problem of tweet engagement
prediction.
It aimed to examine whether deep neural networks are suitable for this task, 
especially concerning model performance and practical applicability.
In short, developed models were structured to produce both retweet and favorite
estimates for any given tweet from the public Twitter API.
Additionally, a special emphasis was put on production-readiness of all
developed deep neural networks, marked by the ability to deliver near real-time
predictions.
Thus, feature engineering and preprocessing were reduced to a minimum and no
additional data was added to the raw tweet objects.

Four distinct data sets were compiled in order to test deep learning methods.
Examined tweets came from three separate user groups, namely celebrities, US
politicians and companies.
Three specialized data sets contained eight months of tweets for each of the user
groups, whereas the larger, combined data set consisted of tweets from all
user groups over the course of two full years (2016-17).
All tweets were more than two months old at the time of collection, guaranteeing
the retweet process to be sufficiently completed.
Furthermore, this thesis was focussed on developing models for English tweets.
As inputs to the deep neural networks, structured and unstructured features
were extracted from all tweets.
Structured features comprised network, activity and content statistics, whereas
unstructured features solely consisted of the actual tweet content.
Three model types, namely linear models, deep feedforward and multi-input
neural networks, were trained on all data sets.
Here, the first two models rely on structured features only, whereas the multi-input
models use both features types.
All models were trained on the tasks of multi-class classification, i.e.,
predicting engagement intervals, and regression, i.e., predicting a positive, real-valued
number.

Experiments conducted in this thesis show that deep feedforward networks
give better results than simple linear models, independent of data set and task.
The increase is most probably caused by improved representational capability
as a result of adding more layers of abstraction.
Adding tweet texts as further inputs only improves performance for larger data
sets.
On smaller data sets up to 100,000 examples, the word corpus from all tweets
is presumably too small for the networks to learn powerful features from it.
Nevertheless, performance improvements for larger data sets (more than 800,000
training examples) are significant.
More precisely, multi-input neural networks provide a 13\% increase in performance
for the classification task, on the regression task increases in prediction
precision are as high as 27\%.
Overall, classification accuracies of around 70\% were achieved on all data sets.
