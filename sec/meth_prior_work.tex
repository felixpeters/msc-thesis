\subsection{Prior Work}
\label{sec:prior_work}

This section lists previous work in tweet engagement prediction, starting with
general retweet research and then introducing models that cope with different
kinds of prediction problems.
Predictive models in this area are largely focused on retweet prediction.
This can be explained by the sharing effect of retweeting, which can be interpreted
as information diffusion, a highly researched area in social network analysis.
Previously developed models were roughly classified according to their problem type.

\subsubsection{General retweet research}

General research on the act of retweeting offers intuitions about the mechanism
and gives way for feature ideas for machine learning models.
Boyd et al.~\cite{Golder} examine conversational aspects of retweeting and
compare the mechanism to other behavioral conventions on the Twitter platform,
namely mentions and hashtags.
Where mentions constitute a form of direct messaging and hashtags enable
communication around specific topics, retweeting is a form of information
diffusion and participating in conversations without actively contributing.
Furthermore, the authors identify motivations for retweeting, e.g., showing
presence as a listener or validating a tweet.
According to their examination, the most commonly retweeted content is
time-sensitive, e.g., breaking news.
Factors influencing retweetability have also been the subject of other work.
Features are often separated into content and contextual features.
Among content features, the use of URLs and hashtags has been shown to be
positively correlated with retweetability, especially for URLs that are seen
as particularly interesting by human respondents~\cite{Suh}.
Contextual features comprise all factors related to the source user, his
following and activitiy on the platform.
The number of followers and friends, as well as account age have positive
effects on retweetability.
Moreover, past activity, e.g., tweet frequency was shown to not be significantly
influential~\cite{Bakshy2011, Suh}.
Retweeting has also been analyzed in temporal fashion, mainly through the use
retweet graphs which constitute representations of cascades.
Kwak et al.~\cite{Kwak2010} show that most retweeting acts occur in the first
and second degree network of the source user.
Additionally, most retweeting happens in a concise time frame.
Namely, 50 \% of retweets take place within one hour, 75 \% within a day and more
than 90 \% in one month after tweet creation.

\subsubsection{Retweet probability prediction}

Retweet probability prediction refers to the problem of forecasting the
probability that a given destination user will retweet a specific tweet of
the source user.
This problem has drawn interest from the research community in the past, and
multiple models have been developed for this purpose.
First models only used very simple features like tweet words, identity of the
source user and the number of his followers and friends~\cite{Zaman2010}.
Other work focuses on content features, which can be divided into two separate
categories.
Low-level features include used words, the number of hashtags, URLs and mentions,
as well as usage of positive and negative terms according to predefined word 
classification.
High-level features on the other hand describe more complex features such as
the association to topics or sentiments.
Naveed et al.~\cite{Naveed2011} develop a logistic regression model using
both types of features and find that tweets regarding more general and 
negatively connoted topics have a higher probability of being retweeted.
More complex features are included by Peng et al.~\cite{Peng2011},
who employ conditional random fields for their predictive model.
The authors include content features such as topic similarity of a tweet with 
interests of the source user and his audience.
In addition, activity data is used, e.g., tweet frequencies of user and
followers around specific times.
A new set of features involves the relationship between source and destination
user, i.e., the number of mutual followers, friends and mentions.
It is shown that user features are most helpful for the prediction of
retweet probability.
Lee et al.~\cite{Lee2014} go a step further and try to identify the most
effective message propagator for a given tweet.
They rely on a rich set of manually crafted features around personality and activity data
and employ different machine learning models, e.g., random forest, logistic
regression and support vector machines.

\subsubsection{Classification approaches}

As introduced in ch.~\ref{sub:dl_terminology}, classification models learn a
mapping between inputs and predefined sets of outputs.
A model employing an output variable with exactly two possible values is referred
to as a \textit{binary classification model}.
This approach can be used for retweet prediction approach, namely if the goal
of the model is to simply determine whether a given tweet will be retweeted
or not.
So-called \textit{multi-class classification models} aim to deliver a more
precise prediction regarding the number of retweets.
Hong et al.~\cite{Hong2011} have developed both types of models in their work
using a logistic regression approach.
Whereas the binary classification model shows promising results, the multi-class
model only performs well for predicting retweet counts of 0 and larger than
10,000.
Classes inbetween these two extremes are predicted with much lower accuracy.
Another binary classification model was introduced by Petrovic et al.~\cite{Petrovic2011}
who focus on the creation a deployable model.
This leads to the utilization of simple content and contextual features, as described
in the above paragraph.
They find that contextual features, especially regarding the source user,
have higher predictive power than tweet features.
The models achieve accuracy metrics between 70 and 80 \% on different tasks,
which is not significantly different from recorded human performance.

\subsubsection{Regression approaches}

Regression models aim to predict a specific value for given inputs.
For retweet prediction this refers to forecasting a numerical value for the
eventual retweet count.
Since the retweet process after tweet creation has no preset determination date, general assumptions
about the temporal retweet distribution have to be made when building
regression models.
Kupavskii et al.~\cite{Kupavskii2012} try to predict the eventual size of the
retweet cascade, i.e., the graph whose vertices represent retweeting users.
At the time, this constituted a novel task which enables two distinct predictions:
one at the initial point of tweet creation, the second after a short
time of observing the retweet cascade.
Obviously, the second prediction can be made using more information about the
initial message spread.
The authors employ a gradient boosted decision tree based on standard content
features (see above paragraphs).
Additionally, they take retweet ratios of creating user and followers into
account and precalculate PageRank statistics for all considered users on a
separate data set of retweet graphs.
Thus, they gain more insights into retweeting behavior of users in their data
set.
As expected, observing retweet cascades for some time improves the quality
of their prediction considerably.
A comparable approach is used by Zaman et al.~\cite{Zaman2014}, whose work
examines retweet time series for 52 manually chosen tweets.
They find that the median of all retweets occurs between four minutes and three
hours after tweet creation.
In addition, they develop a Bayesian model for evaluating the retweet graph
which is able to utilize parts of the time series data to improve its
prediction.
The model outperforms a linear regression baseline using only
the follower count as input data.
Moreover, the prediction becomes more precise the more time series data is utilized
as an input.

Having established this line of prior research, the approaches used in this
work can now be described.
