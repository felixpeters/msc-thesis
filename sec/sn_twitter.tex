\subsubsection{Twitter}
\label{sub:sn_twitter}

Twitter is a microblogging service that enables its users to share short
messages with their followers in more than 40 languages.
The company was founded in 2006 and has grown to 330 million monthly active
users worldwide as of 2017.
Main revenue source for Twitter are advertisements which account for more than
85 \% of revenue in 2017.
The last fiscal quarter of 2017 marked the first profitable period in Twitter's
history, with profits equalling 91 million USD at a revenue of 732 million USD.
Here, video is main advertisement format used on the platform~\footcite{Twitter2018}.
Twitter has also invested in live streaming events, e.g., football games~\footnote{\url{https://www.bloomberg.com/news/articles/2016-04-05/twitter-said-to-win-nfl-deal-for-thursday-night-streaming-rights}}.

Later parts of this thesis relate to specific Twitter terminology which will be 
presented in the following.
Users of the Twitter platform open an account with a unique user name (starting
with the `@' symbol).
Users can be verified by Twitter if the accounts is of public interest, e.g.,
for celebrities, companies or politicians~\footnote{\url{https://help.twitter.com/de/managing-your-account/about-twitter-verified-accounts}}.
If a user wants to receive updates from another user, he can follow his
account.
Relations between users are not necessarily symmetric, so that incoming relations,
i.e., being followed, are called `followers' and outgoing relations, i.e.,
following, are refered to as `friends'.
Twitter allows curating users onto lists, which enables receiving topic-bound
status updates.
The number of lists a user is assigned to can be found in his account under the
term `listings'.
Users publish short messages called `tweets', which can be seen in their user
account and the timeline of followers.
Other users can engage with tweets by through `favorites', `retweets' and `replies'.
When sharing a tweet via a retweet, users have the option to add extra content.
If they opt for this approach, the resulting update is treated like a whole new
status message and referred to as a `quote'~\footcite{Kwak2010}.
Quotes receive distinct engagement statistics, whereas engagement in retweets
is counted towards the original message.
The above described differentiation is important for later parts of this thesis.

This section about Twitter terminology completes the background chapter.
The next chapter will explain methodology of experiments in this work.
