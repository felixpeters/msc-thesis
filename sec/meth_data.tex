\section{Data}
\label{sec:data_collection}

The main goal of this thesis regarding the examined data is to have diverse
data sets from users for which engagement prediction has practical relevance.
The diversity requirement mainly comprises different retweet distributions
so that the model performance can evaluated in a generalized manner.
Predicting the popularity of tweets is relevant for all individuals and
organizations that somehow rely on social media marketing or customer support.
Therefore, the collected data sets include data from public figures and
companies, most of whom actively maintain a Twitter account.
The selected accounts were separated into three groups and saved as Twitter lists
in the author's profile (see Appendix).
For each of the three groups, all tweets from January to August 2017 were
fetched from the public Twitter API~\footnote{\url{https://developer.twitter.com/en/docs/tweets/timelines/api-reference/get-statuses-user_timeline.html}}.
Afterwards, retweets that the specified users made were excluded since their
engagement statistics are not related to their accounts, but the users who
originally created their tweets.
In summary, the final data sets include all status updates and quotes from the
specified users in the given time frame.
Since the data was collected in October/November 2017, all tweets in the data
set are older than the threshold of one month that was assumed in the previous
subsection.
The following paragraph will introduce the three distinct user groups in more
detail.

\begin{table}
\centering
\begin{tabular}{lllr}
\toprule
Name & Accounts & Number of accounts & Total tweets \\
\midrule
Companies & Fortune 500 companies & 441 & 254,059 \\
Politicians & US Governors and Senators & 162 & 94,919 \\
Celebrities & Athletes, actors, actresses \& celebrities & 170 & 46,986 \\
\bottomrule
\end{tabular}
\caption{Summary of collected data sets}
\label{tab:dataset_summary}
\end{table}

Tab.~\ref{tab:dataset_summary} sums up all data sets used for the purpose of
this work.
The first data set comprises all Twitter accounts from \textit{Fortune 500}
companies in 2017~\footnote{\url{http://fortune.com/fortune500/}}.
All in all, 441 out of 500 companies operate a Twitter presence, which accounted
for a total of 254,059 tweets in the evaluated period.
The second user group includes all Twitter accounts of US Senators~\footnote{\url{https://twitter.com/cspan/lists/senators}} and Governors~\footnote{\url{https://twitter.com/cspan/lists/governors}}
as of October 2017 which were derived from Twitter lists by the television
network \textit{C-SPAN}.
From January to August 2017, 162 politicians created a total of 94,919 tweets.
The smalles data sets stems from celebrities, namely the \textit{Forbes Highest-Paid
Athletes 2017}~\footnote{\url{https://www.forbes.com/sites/kurtbadenhausen/2017/06/15/full-list-the-worlds-highest-paid-athletes-2017}}, \textit{Forbes Highest-Paid Actors and Actresses 2017}~\footnote{\url{https://www.forbes.com/sites/natalierobehmed/2017/08/22/full-list-the-worlds-highest-paid-actors-and-actresses-2017}} and
\textit{Forbes Highest-Paid Celebrities 2017}~\footnote{\url{https://www.forbes.com/sites/zackomalleygreenburg/2017/06/12/full-list-the-worlds-highest-paid-celebrities-2017}}.
The derived data set contains 46,986 tweets from 170 accounts.

Descriptive statistics about all three data sets can be found in ch.~\ref{sec:dda}.
This section concludes the methodology chapter.
The upcoming chapter will describe results of the undertaken experiments for
this thesis.
