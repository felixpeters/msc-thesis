\subsection{History}
\label{sub:dl_history}

After establishing the basic terminology for this thesis, 
it is now possible to take a closer look at the algorithm group of 
\textit{neural networks}. Therefore, this subsection will focus on historical
developments of neural networks, a group of algorithms that deep learning falls
into.
Understanding the origins of this methodology will be helpful in the next 
subsection which explains the more formal aspects of neural networks.
Overall, the history of neural network research is often separated into three
distinct phases of active research and progress in the literature~\cite{Goodfellow2016}.
All three phases will be described in the following.

\paragraph{Phase 1: First models}

The first forms of neural networks were developed in the 1940s, when
researchers tried to model learning machines inspired by the human brain.
The first such model was the McCulloch-Pitts neuron which constituted a linear
model:

\begin{equation}
  f (x, w) = w_1 x_1 + w_2 x_2 + \cdots + w_m x_m
\end{equation}

The neuron was able to recognize two different categories by determining
whether $f(x, w)$ was positive or negative~\cite{McCulloch1943}.
As a downside, there was no way to set the weights via an algorithm.
Instead, the weights had to be set by a human.
The first model of a neuron which overcame this obstacle was the \textit{perceptron}
developed by Rosenblatt~\cite{Rosenblatt1958}. The main contribution of the
perceptron was a simple algorithm to update weights iteratively.
The only requirement for the algorithm to run was a set of labeled examples.
In the same timespan, Widrow and Hoff~\cite{Widrow1960} came up with a similar
model called \textit{adaptive linear element (ADALINE)} for regression problems.
The main disadvantage that could not be overcome at the time lied in the 
linearity of the model which limited its representational capability.
For example, the model from equation 2.1 is not able to learn the XOR function 
(see equations 2.2 to 2.5)~\cite{Minsky1969}.

\begin{align}
  f ([0, 1], w) = 1\\
  f ([1, 0], w) = 1\\
  f ([0, 0], w) = 0\\
  f ([1, 1], w) = 0
\end{align}

These limitations caused a decline in research interest and thus mark the end
of the first phase of neural network research, which is often referred to
as \textbf{cybernetics}~\cite{Goodfellow2016}.

\paragraph{Phase 2: Backpropagation and connectionism}

The second wave of neural network research largely took place from the late
1970s until the mid-1990s. 
The problem of limited representational capability was overcome by wrapping
neurons inside a non-linear function, which basically transforms its output~\cite{Fukushima1975}.
This idea was combined with distributing the representational capacity over
more neurons, thus creating the first \textit{neural networks}. 
The neurons were combined in layers which were connected to each other.
This is also the main reason why this phase of neural network research is
often referred to as \textbf{connectionism}~\cite{Goodfellow2016}.
The intuition behind this was that more neurons would be more capable of
representing the input by acting together. 
Here, each neuron stands for a distinct feature, e.g., the color of an object
or the type of object itself~\cite{Rumelhart1986a}.
Training these more complex models became easier and faster with the
introduction of the backpropagation algorithm~\cite{Rumelhart1986}.
More complex neural networks meant more weights which needed to be updated
iteratively. 
The backpropagation algorithm is still the dominant approach for training
neural networks, because it offers a way of efficiently calculating the
updated weights.
The algorithm will be explained in more detail in the following subsection.
Despite the fast progress in academic research, the interest in neural networks
once again declined in the mid-1990s. The main reasons for this were the
lack of computational power to train deeper models containing more layers, as 
well as the advancements of other fields of machine learning (e.g., Support
Vector Machines) around this time~\cite{Goodfellow2016}.

% Third wave ("deep learning"):
% Training deep networks becomes efficient (Hinton 2006)
% Applications in many areas become possible (to be continued in Recent Developments)

\paragraph{Phase 3: Deep learning}

The difficulties of training deeper neural networks were overcome in 2006.
It was shown that a combination of more efficient training algorithms and
more advanced hardware enabled training deep networks in a reasonable amount of 
time~\cite{Hinton2006, Holden2006}.
Further developments in the techniques and availability of more computational
power led to popularization of deep neural networks and the term \textbf{deep
learning} was coined~\cite{Goodfellow2016, LeCun2015}.
Deep learning was used in models that outperformed other machine learning
algorithms in areas like image and speech recognition~\cite{Krizhevsky2012, Hinton2012}. 
These made use of neural network architectures like \textit{convolutional neural networks} 
and \textit{recurrent neural networks}, that were originally developed during 
the second wave of neural network research~\cite{Fukushima1980, Hochreiter1997, LeCun1998}.

Subsection 2.1.4 will go into more detail about the recent developments of
deep learning techniques. The following subsection will describe the basic
structure, elements and algorithms of neural network architectures in more
detail. This will ease the understanding of the more modern and complex
architectures and lay a mathematical foundation for the later chapters.
